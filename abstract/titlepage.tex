\documentclass[main.tex]{subfiles}
\begin{document}
\begin{titlepage}
  \begin{center}
    {Graph Theory and Applications\\}
    \LARGE{Graph based carpooling recommendation system\\}
    \horrule{0.4pt}
    \large{Gaurav Juvekar} - \large{111408024}\\
    \large{K Y Sandeep Somanna} - \large{111408025}\\
    \large{Aarti Kashyap} - \large{111408070}\\
    \normalsize{\today{}}
  \end{center}
  \horrule{1pt}
  \begin{abstract}
    A carpooling recommendation system can alleviate the problems of traffic
    congestion and environmental pollution effectively in big cities. The system
    partitions commuters travelling to the same destination based on vehicle
    capacities, their periodic travel times and their mutual preferences to
    share rides with each other. If the set of drivers is known in advance, then
    for any vehicle capacity, the problem is equivalent to the assignment
    problem in bipartite graphs. Otherwise, when we do not know in advance who
    will drive their vehicle and who will be a passenger, the problem is
    NP-hard. The road distances between the set of commuters and from commuters
    to the major roads are modeled as a directed graph which acts as a
    simplified map of the city layout. In this project, we aim to develop a
    proof of concept to solve a restricted version of the problem limited to a
    single common destination.
  \end{abstract}
\end{titlepage}
\end{document}
